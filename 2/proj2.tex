\documentclass[a4paper, 11pt, twocolumn]{article}
\usepackage[left=1.4cm, top=2.3cm, text={18.2cm, 25.2cm}]{geometry}
\usepackage[czech]{babel}
\usepackage[IL2]{fontenc}
\usepackage{times}
\usepackage[utf8]{inputenc}
\usepackage{alltt, amsthm, amsmath, amsfonts}

\newtheorem{definice}{Definice}

\newtheorem{sentence}{Věta}

\begin{document}

\pagenumbering{arabic}

\begin{titlepage}
    \onecolumn
    \begin{center}
        {\Huge\textsc{Vysoké učení technické v~Brně\\[0.5em]}}
        {\huge\textsc{Fakulta informačních technologií}}\\
        \vspace{\stretch{0.382}}
        {\LARGE Typografie a publikování\,--\,2. projekt\\[0.4em] 
        Sazba dokumentů a matematických výrazů}\\
        \vspace{\stretch{0.618}}
    \end{center}
    {\Large 2023 \hfill Ivan Mahút (xmahut01)}
\end{titlepage}

\clearpage
\setcounter{page}{1}

\section*{Úvod} 
\label{uvod}
V~této úloze si vyzkoušíme sazbu titulní strany, matematických vzorců, 
prostředí a dalších textových struktur obvyklých pro technicky zaměřené texty\,--\,například Definice \ref{def:1} nebo rovnice \eqref{eq:3} na straně \pageref{def:1}. Pro vytvoření těchto odkazů 
používáme kombinace příkazů \verb|\label|, \verb|\ref|, \verb|\eqref| a \verb|\pageref|. Před odkazy patří nezlomitelná mezera. 
Pro zvýrazňování textu jsou zde několikrát použity příkazy \verb|\verb| a \verb|\emph|.\par 
Na titulní straně je použito prostředí \verb|titlepage| a sázení nadpisu podle optického středu 
s~využitím \emph{přesného} zlatého řezu. Tento postup byl probírán na přednášce. 
Dále jsou na titulní straně použity čtyři různé velikosti písma a mezi dvojicemi řádků textu je použito odřádkování se zadanou relativní 
velikostí 0,5\,em a 0,4\,em\footnote[1]{Nezapomeňte použít správný typ mezery mezi číslem a jednotkou.}.

\section{Matematický text}
\label{sec:1}
V~této sekci se podíváme na sázení matematických symbolů a výrazů v~plynulém textu pomocí prostředí \texttt{math}. 
Definice a věty sázíme pomocí příkazu \verb|\newtheorem| s~využitím balíku \texttt{amsthm}. 
Někdy je vhodné použít konstrukci \verb|${}$| nebo \verb|\mbox{}|, která říká, že (matematický) text nemá být zalomen.
\begin{definice} 
\label{def:1}
    \emph{Zásobníkový automat} (ZA) je definován jako sedmice tvaru ${A =(Q,\Sigma,\Gamma,\delta,q_{0},Z_{0},F)}$, kde:
    \begin{itemize}

        \item $Q$ je konečná množina \textup{vnitřních (řídicích) stavů},
        \item $\Sigma$ je konečná \textup{vstupní abeceda},
        \item $\Gamma$ je konečná \textup{zásobníková abeceda},
        \item $\delta$ je \textup{přechodová funkce} $Q\times(\Sigma\cup \{ \epsilon \} )\times\Gamma\rightarrow 2^{Q\times\Gamma^{*}}$,
        \item ${q_0 \in Q}$ je \textup{počáteční stav,} $Z_{0}\in\Gamma$ je \textup{startovací symbol zásobníku} a ${F \subseteq Q}$ je množina \textup{koncových stavů.}
    
    \end{itemize}
\par
\textup{Nechť} $P = (Q, \Sigma, \Gamma, \delta , q_{0}, Z_{0}, F )$ \textup{je ZA}. Konfigurací \textup{nazveme trojici} 
$(q, w, \alpha) \in Q \times \Sigma^* \times \Gamma^*$\textup{, kde $q$ je aktuální stav vnitřního řízení, 
$w$ je dosud nezpracovaná část vstupního řetězce a} 
$\alpha = Z_{i_{1}} Z_{i_{2}} \dots Z_{i_{k}}$ \textup{je obsah zásobníku.}
\end{definice}


\subsection{Podsekce obsahující definici a větu}
\begin{definice} 
\label{def:2}
\textup{Řetězec} $ w $ \textup{nad abecedou} $\Sigma$ \textup{je přijat ZA} $A$ jestliže ${(q_0, w, Z_0) \underset{A}{\overset{*}{\vdash}} (q_{F},\epsilon, \gamma) }$ pro nějaké $ \gamma \in \Gamma^* $ a ${q_F \in F}$.
Množina ${L(A)=\{w \mid w}\mbox{ je přijat ZA }A\}\subseteq\Sigma^*$ je \textup{jazyk přijímaný ZA $ A $.}
\end{definice}

\begin{sentence} 
    Třída jazyků, které jsou přijímány ZA, odpovídá \textup{bezkontextovým jazykům.}
\end{sentence}

\section{Rovnice}
\label{sec:2}
Složitější matematické formulace sázíme mimo plynulý text pomocí prostředí \verb|displaymath|. Lze umístit i několik výrazů na jeden řádek, ale pak je třeba tyto vhodně oddělit, například příkazem \verb|\quad|.
\begin{displaymath}
    1^{2^{3}} \neq \Delta^{1}_{\Delta^{2}_{\Delta^3}} \quad
    y^{11}_{22} - \sqrt[9]{x + \sqrt[7]{y}} \quad
    x > y_1 \leq y^2
\end{displaymath}
V~rovnici \eqref{eq:2} jsou využity tři typy závorek s~různou explicitně definovanou velikostí. 
Také nepřehlédněte, že nasledující tři rovnice mají zarovnaná rovnítka, a použijte k~tomuto účelu vhodné prostředí.
\begin{eqnarray}
    - \cos^2 \beta & = & \frac{\frac{\frac{1}{x} + \frac{1}{3}}{y} + 1000}{\prod\limits _{j=2}^8 q_j} \label{eq:1}\\
    \left( \Bigl\{ b \star \bigl[3 \div 4\bigr] \circ a \Bigr\}^{\frac{2}{3}}\right) & = & \log_{10} x \label{eq:2}\\
    \int_{a}^{b} f(x)\,\mathrm{d}x & = & \int_{c}^{d} f(y)\,\mathrm{d}y \label{eq:3}
\end{eqnarray} 
V~této větě vidíme, jak vypadá implicitní vysázení limity ${\lim_{m\to \infty} f(m)}$ v~normálním odstavci textu. 
Podobně je to i s~dalšími symboly jako ${\bigcup_{N \in \mathcal{M}}N}$ či ${\sum_{i=1}^{m} x_{i}^2}$.
S~vynucením méně úsporné sazby příkazem \verb|\limits| budou vzorce vysázeny v~podobě ${\lim\limits _{m \to \infty} f(m)}$ a ${\sum\limits _{i=1}^{m} x_{i}^4}$.

\section{Matice}
\label{sec:3}
Pro sázení matic se velmi často používá prostředí \verb|array| a závorky (\verb|\left|, \verb|\right|). 
\[ \mathbf{B} = \left| \begin{array}{cccc}
        b_{11} & b_{12} & \cdots & b_{1n}\\
        b_{21} & b_{22} & \cdots & b_{2n}\\
        \vdots & \vdots & \ddots & \vdots\\
        b_{m1} & b_{m2} & \cdots & b_{mn}\\
        \end{array} \right|
        = \left| \begin{array}{cc}
        t & u\\
        v~& w\\ 
        \end{array} \right|
        = tw - uv
\]

\begin{displaymath} 
    \mathbb{X} = \mathbf{Y} \Longleftrightarrow 
    \left[\begin{array}{ccc}
        \phantom{} & \Omega + \Delta & \hat{\psi}\\
        \vec{\pi} & \omega & \phantom{}\\
    \end{array}\right]
    \neq 42 
\end{displaymath}
\indent Prostředí \verb|array| lze úspěšně využít i jinde, například na pravé straně následující rovnice. 
Kombinační číslo na levé straně vysázejte pomocí příkazu \verb|\binom|.
\begin{displaymath}
    \binom{n}{k} = \left\{
        \begin{array}{c l}
        0 & \text{pro } k < 0 \\
        \frac{n!}{k!(n-k)!} & \text{pro } 0 \leq k \leq n \\
        0 & \text{pro } k > 0
        \end{array} \right.
\end{displaymath}

\end{document}

