\documentclass[a4paper, 11pt]{article}
\usepackage[left=2cm, top=3cm, text={17cm, 24cm}]{geometry}
\usepackage[czech]{babel}
\usepackage{times}
\usepackage{hyperref}
\usepackage[utf8]{inputenc}

\begin{document}
\pagenumbering{arabic}

\begin{titlepage}
    \begin{center}
        \Huge\textsc{Vysoké učení technické v~Brně}\\
        \huge\textsc{Fakulta informačních technologií}\\
        \vspace{\stretch{0.382}}
        \LARGE Typografie a publikování\,--\,4. projekt\\ 
        \Huge Bibliografické citace\\
        \vspace{\stretch{0.618}}
    \end{center}
    {\Large \today \hfill Ivan Mahút}
\end{titlepage}

\pagebreak
\setcounter{page}{1}

\section{Historie}
Vynález Johannese Gutenberga, tiskařský lis, se označuje jako začátek typografie v~Evropě. Následovalo několik staletí zdokolanování tiskařských stojů. 
Z~technického hlediska nastal průlomový bod ve 20. století a začátkem používání počítačů \cite{jirasek}.

\subsection{Tiskové písmo}
Prvním počítačem, který podporoval aplikace na sázení byl Apple Macintosh, typografové byli nuceni se naučit pracovat s~moderními technologiemi. Následující desetiletí byl nejpoužívanější program QuarkXpress.
Disponoval podporou českého jazyka což zahrnovalo fonty, dělení slov \cite{vasilova}.

\section{Význam typografie}
\begin{quotation}
\textit{Typografie dává přinejmenším dva druhy smyslu, pokud má vůbec nějaký smysl.}
\end{quotation}
Typografie může být pro běžného člověka skryta, má však historický a vizuální význam. Vizuální stránka je viditelná i pro běžného člověka, ale historické rukopisy jsou přístupné pouze malé skupině lidí. viz \cite{bringhurst}

\subsection{Význam typografie pro vnímání zpráv}
Typografie má podstatný vliv na úspěšnost a význam komunikace. Výběr správného písma, velikosti či stylu je velmi podstatné hladiště ve výsledném obraze napsaného textu.
Odlišná písma v~lidech vyvolávají různé emoce, více na \cite{pilka}.

\subsection{Význam typografie na webu}
V~dnešní době umožňují moderní technologie aktivně reagovat na řadu faktorů. Z~technického hlediska na formát, možnosti zařízení ale také z~hlediska uživatele jako jeho preference nebo vzdálenost čtení.
\begin{quotation}
    \textit{Design již není o~přizpůsobení neměnného obsahu jednomu konkrétnímu provedení; web nás nutí přemýšlet o~typografii z~hlediska parametrů a ujasnit si obsah versus forma.}\cite{indra}
\end{quotation}
Publikování na internetu vyžaduje od autora použití otevřeného formátu. To znamená že při psaní textu vzhled písma není prioritizovaný. Důležité jsou významy znaků které jsou zaznamenány. viz. \cite{janak}

\subsection{Význam typografie v~designu}
Plakáty pro filmy nebo seriály mají vždy velmi propracovanou typografickou stránku. Typografie je v~designu důležitá hlavně z~důvodu zaujetí diváka ale i snadného zapamatování konkrétní značky \cite{gondar}.

\subsection{Význam typografie ve vysokoškolském studiu}
V~rámci vysokoškolského štúdia je vyžadované písanie odborných textov. Odborný text musí spĺňať fakultné normy a zároveň nesmie porušiť zaužívané normy a pravidlá. viz. \cite{knytl}




\pagebreak
\bibliographystyle{czechiso}
\bibliography{proj4}

\end{document}